\documentclass[12pt]{article}
\usepackage[english]{babel}
\usepackage[utf8x]{inputenc}
\usepackage{amsmath}
\usepackage{graphicx}
\usepackage[colorinlistoftodos]{todonotes}
\usepackage{subfig}
\usepackage{makecell}


\begin{document}
\begin{titlepage}
\newcommand{\HRule}{\rule{\linewidth}{0.5mm}} 
\center 
%opening

\textsc{\LARGE ITBA}\\[1.5cm] 
\textsc{\Large Base de datos I}\\[0.5cm] 
\textsc{\large Profesores: Leticia Irene Gomez, Valeria Inés  Soliani, Cecilia  Rodriguez Babino }\\[0.4cm] 

\HRule \\[0.4cm]
{ \huge \bfseries Trabajo Pr\'actico Especial

\HRule \\[1.5cm]
 
\begin{minipage}{0.4\textwidth}
\begin{flushleft} \large
\emph{Alumnos:}\\
Lautaro Pinilla \\
Micaela Banfi \\
Nicolas Paganini \\

\end{flushleft}
\end{minipage}
~
\begin{minipage}{0.4\textwidth}
\begin{flushright} \large
\emph{} \\
57504 \\
57293 \\
12345 \\

\end{flushright}
\end{minipage}\\[2cm]


{\large 13 de Junio de 2019}\\[2cm]

\vfil
\end{titlepage}

\section{Introducci\'on}
En el siguiente trabajo pr\'actico se demostra\'a el uso de conceptos avanzados de SQL, tales como PSM y triggers, vistos a lo largo de la materia. Se trabajar\'a sobre el archivo SalesbyRegion.csv, el cual contiene informaci\'on de ventas de una empresa de Retail. Especificamente debemos migrar los datos del archivo csv a una base de datos, generar un cálculo, producir un reporte y realizar algunas validaciones.




\section{Investigaciones}

\section{Dificultades durante el desarrollo y su soluci\'on}
\subsection{Dificultades en el punto d}

\section{Proceso de importaci\'on de datos realizado}

\section{Roles de cada integrante}
\begin{itemize}
  \item Encargado del informe: Micaela Banfi
  \item Encargado de las funciones: Nicolas Paganini
  \item Encargado del trigger: Lautaro Pinilla
  \item Encargado del funcionamiento global del proyecto: 
  \item Encargado de investigación: 
\end{itemize}
\section{Bibliograf\'ia} 


\end{document}